\documentclass{article}
\usepackage[utf8]{inputenc}
\usepackage{listings}
\usepackage{setspace}
\usepackage{parskip}
\usepackage{geometry}
\usepackage{hyperref}
\usepackage{verbatim}
\usepackage{fontspec}

\geometry{margin=1in}

\title{Windows 10/11 Initial Setup and Configuration}
\author{}
\date{}

% Set font to Consolas
\setmonofont{Consolas}

\begin{document}

\maketitle

\section*{Windows 10/11 Setup}
\begin{itemize}
    \item DO NOT CONNECT TO WIFI OR NETWORK
    \item Run through Setup until you get to network configuration
    \item Press \textbf{Shift+F10} (SHIFT+FN+F10) for Command Prompt
    \item Type: \begin{verbatim}oobe\bypassnro\end{verbatim}
    \item After reboot run through setup, at Network Configuration select "I don't have Internet"
    \item Continue with limited Setup
\end{itemize}

\section*{Local Username Initial Setup}
\begin{itemize}
    \item Username: \texttt{kyma.admin}
    \item Password: \texttt{000000}
    \item Security Questions:
    \begin{itemize}
        \item What was your first pet's name? \texttt{cali}
        \item What's the name of the city where you were born? \texttt{cali}
        \item What was your childhood nickname? \texttt{cali}
    \end{itemize}
\end{itemize}

\section*{User Account Configuration}
\begin{itemize}
    \item Username: \texttt{hans@kymabatteries.com}
    \item Password: \texttt{Tuesday@0227!}
\end{itemize}

\section*{Wi-Fi Configuration}
\begin{itemize}
    \item Network Name: \texttt{Kyma}
    \item Password: \texttt{6s=i87qg5fcp}
\end{itemize}

\section*{Change Computer Name}
\begin{itemize}
    \item Open File Explorer
    \item Right-Click \textbf{This PC}
    \item Properties
    \item Rename this computer
    \item Computer Name: \texttt{KYMA-HANS}
    \item Reboot
\end{itemize}

\section*{Add New User}
\begin{itemize}
    \item Settings
    \item Other Users
    \item Add Account
    \item Username: \texttt{hans}
    \item Password: \texttt{Email Password}
    \item Security Questions:
    \begin{itemize}
        \item What was your first pet's name? \texttt{cali}
        \item What's the name of the city where you were born? \texttt{cali}
        \item What was your childhood nickname? \texttt{cali}
    \end{itemize}
    \item Select Account
    \item Change Account Type: Administrator
    \item Login to new account
    \item Connect to Internet
\end{itemize}

\section*{VPN Installation and Configuration}
Follow the instructions below to install the VPN application for your operating system. Once installed, download the VPN profile attached to this email and double-click on it to add to your OpenVPN or TunnelBlick application.

\subsection*{MAC}
\begin{itemize}
    \item Download and install the Stable Version of Tunnelblick from this link:
    \url{https://tunnelblick.net/downloads.html}
\end{itemize}

\subsection*{PC}
\begin{itemize}
    \item Download the Windows 64 Bit OpenVPN from this link:
    \url{https://openvpn.net/community-downloads/}
\end{itemize}

To start the Windows OpenVPN connection, right-click the icon in the notification area, then click connect. To start the Mac Tunnelblick VPN, click the icon in the Menu Bar, then select the KB-VPN configuration. The VPN connection will prompt you for your username and password; check the box to save the password.

\begin{itemize}
    \item Username: Your first name
    \item Password: \texttt{Kyma@2024!}
\end{itemize}

This will also be your username for the new server.

\subsection*{User Remote Desktop to Access the Servers}
\begin{itemize}
    \item ZMUDOWSKI: \texttt{192.168.1.10}
    \item YOSEMITE: \texttt{192.168.1.12}
\end{itemize}

\subsection*{Admin Credentials for the New Server, YOSEMITE}
\begin{itemize}
    \item Username: \texttt{kyma.admin}
    \item Password: \texttt{\#KBatt@2024!}
\end{itemize}

\section*{Applications}
\begin{itemize}
    \item Remove any Anti Virus or junk apps
    \item Download Google Chrome
    \item Download Adobe Acrobat DC Reader
    \item Login to \texttt{office.com} and install Office
    \item Download Teams for Work or School from browser
    \begin{itemize}
        \item When downloading teams (if there is trouble installing):
        \begin{itemize}
            \item Click "Problems installing new teams?"
            \item Click "install" under App Installer
            \item In Microsoft Store click "update"
            \item Wait for download and close store when complete
            \item Go back to teams download page and install
        \end{itemize}
    \end{itemize}
    \item Sign in to Activate Office
    \item Setup Outlook
    \item Sign in to OneDrive
    \begin{itemize}
        \item Enable Backup
    \end{itemize}
    \item Sign in to Teams
    \item Install any Additional Applications
    \item Set Default Apps
    \begin{itemize}
        \item Chrome
        \item Outlook
        \item Adobe
    \end{itemize}
\end{itemize}

\section*{Clean Up Task Bar}
\begin{itemize}
    \item Remove Edge
    \item Remove Chat
    \item Remove Store
    \item Pin Outlook
    \item Pin Teams
    \item Pin Chrome
\end{itemize}

\section*{Add Desktop Icons}
\begin{itemize}
    \item Right-Click Desktop
    \item Personalization
    \item Themes
    \item Desktop Icons
    \begin{itemize}
        \item Computer
        \item User's Files
    \end{itemize}
    \item Right-Click Desktop
    \item View
    \item Auto Arrange
\end{itemize}

\section*{Disable Sleep}
\begin{itemize}
    \item Click Start
    \item Power
    \item Edit Power Plan
    \item Put the computer to sleep: \textbf{Never}
\end{itemize}

\section*{Check for Updates After Setting Up Computer}
\begin{itemize}
    \item In taskbar, type "update" in search bar
    \item Select "check for updates"
    \item Download and install any windows updates
    \item Restart if needed
\end{itemize}

\end{document}
